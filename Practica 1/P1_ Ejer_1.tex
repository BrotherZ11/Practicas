\documentclass[12pt]{article}

\usepackage{lmodern}
\usepackage[T1]{fontenc}
\usepackage[spanish,activeacute]{babel}
\usepackage{mathtools}
\usepackage{graphicx}


\title{Práctica 1}
\author{David Zarzavilla Borrego}
\date{Curso 2022/23}

\begin{document}

\maketitle

\section{Introduction}
En este primer ejercicio queremos hacer la potencia $R^3$ de 

R=\{(1,1),(1,2),(2,3),(3,4)\}

\vspace{5mm}
Para hacer $R^3$ primero tenemos que hacer $R^2$, es decir:

$R^2=R \circ R$

$R^3 = R^2 \circ R$

Para calcular $R^2$ usamos la propiedad transitiva:

$(a,b) \in R \wedge (b,c) \in R \rightarrow (a,c) \in R$, siendo 

R =\{(1,1),(1,2),(2,3),(3,4)\}

\vspace{5mm}
Empezamos con $R^2$

$R^2$ = \{(1,1),(1,2),(1,3),(2,4)\} ya que al usar la propiedad transitiva, el 1 al estar con el 2, también esta con el 3, ya que el 2 está con el 3, así como el 2 esta con el 4, porque el 2 está con el 3 y el 3 con el 4.

\vspace{5mm}
Finalmente calculamos $R^3$ a partir de $R^2$

$R^3$ = \{(1,1),(1,2),(1,3),(1,4)\} ya que el único número que está con el 4 es el 2 y como el 2 está con el 1, el 1 está con el 4.

\newpage
Ahora, tenemos que comprobar la solución con el script powerrelation.m

\vspace{5mm}

\includegraphics[scale=0.25]{Captura.png}

\vspace{5mm}
Como podemos observar en la captura realizada en el terminal octave y realizando el script, las soluciones son las mismas.
\end{document}
