\documentclass[12pt]{article}

\usepackage{lmodern}
\usepackage[T1]{fontenc}
\usepackage[spanish,activeacute]{babel}
\usepackage{mathtools}
\usepackage{graphicx}


\title{Práctica 2}
\author{David Zarzavilla Borrego}
\date{Curso 2022/23}
\begin{document}

\maketitle

\section{Ejercicio 1}
\subsection{1a}

En este primer ejercicio vamos a crear una AFD que reconoza las cadenas que solo contienen a.

Un AFD es una 5-tupla (K,$\sum, \delta$, s, F)

K es un conjunto de estados no vacíos

$\sum$ es un alfabeto

$\delta$ : K x $\sum$ $\rightarrow$ K es la funcion de transicion

s $\in$ K es el estado inicial

F $\subseteq$ K es el conjunto de estados finales

\vspace{5mm}
En este apartado nos pide un ejemplo para que reconozca la cadena pedida, por tanto:

$M = (\{q_0, q_1\}, \{a, b\}, \{(q_0, a, q_0), (q_0, b, q_1), (q_1, a, q_1), (q_1, b, q_1)\}, q_0, \{q_0\})$

\vspace{5mm}
\includegraphics[scale=1]{DFA.png}

\subsection{1b}

En este apartado nos piden probar 6 cadenas, para ver si son aceptadas o rechazas, para ello usamos el programa JFlap, quedando tal que así:

\vspace{5mm}
\includegraphics[scale=0.35]{test.png}

\newpage
\section{Ejercicio 2}

En este ejercio usaremos el script finiteautomata.m para crear el automata que hemos diseñado, para ello primero tendremos que introducir en finiteautomata.json nuestro automata y poder ejecutarlo con el script en octave, al ejecutarlo nos da lo siguiente:

\begin{verbatim}
    {
    "name" : "a*",
    "representation" : {
      "K" : ["q0", "q1"],
      "A" : ["a", "b"],
      "s" : "q0",
      "F" : ["q0"],
      "t" : [["q0", "a", "q0"],
             ["q0", "b", "q1"],
             ["q1", "a", "q1"],
             ["q1", "b", "q1"]]
      }
\end{verbatim}

\vspace{5mm}
\includegraphics[scale=0.4]{AF.png}

\vspace{5mm}
Pudiendo poner el resultado escrito a mano en LaTeX:

$M = (\{q_0, q_1\}, \{a, b\}, \{(q_0, a, q_0), (q_0, b, q_1), (q_1, a, q_1), (q_1, b, q_1)\}, q_0, \{q_0\})$

$w = a$

$(q_0, a) \vdash (q_0, \varepsilon)$

x $\in$ L(M)

ans = 1

\end{document}
